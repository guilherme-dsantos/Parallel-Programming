\documentclass[11pt]{report}
\title{Assignment 2: Coin Counter}
\author{Guilherme Santos \texttt{fc62533}}
\date{\today}

\usepackage[left=1in,right=1in,top=1in,bottom=1in]{geometry}
\usepackage{tikz,pgfplots}
\usepackage{subfig}
\usepackage{caption}
\usepackage{amsmath}
\begin{document}

\maketitle

What was the parallelization strategy chosen? Used fork join because it is a recursive algorithm (incomplete)
What results did you achieve? do the graphs and s..
What fine-tuning did you do to improve the performance? Started by using the RecursiveTask.getSurplusQueuedTaskCount() > 2
and then used depth >= and achived better results with the latest.

\begin{figure}[h]
  
    \caption*{Execution time with \texttt{coins.length=30} and \texttt{LIMIT=999}}
   
    \begin{minipage}{0.5\textwidth}
      \centering
      \begin{tikzpicture}
        \begin{axis}[xlabel={Threads}, ylabel={Time (s)}, xtick={2, 8, 16, 24}]
          \addplot+[only marks, scatter, mark size=2.9pt]
          coordinates {
            (2,32.89)
            (2,32.81)
            (2,33.00)
            (2,32.51)
            (2,32.81)
            (8,11.59)
            (8,11.91)
            (8,12.00)
            (8,11.72)
            (8,12.09)
            (16,11.33)
            (16,10.89)
            (16,10.59)
            (16,10.69)
            (16,10.82)
            (24,10.60) 
            (24,10.34) 
            (24,10.52) 
            (24,10.82) 
            (24,10.48)
          };
        \end{axis}
      \end{tikzpicture}
      
      \caption{Parallel}
    \end{minipage}%
    \hspace{2em}
    \begin{minipage}{0.5\textwidth}
      \centering
      \begin{tikzpicture}
        \begin{axis}[xlabel={Coins}, ylabel={Time (s)}, xtick={30}]
          \addplot+[only marks, scatter, mark size=2.9pt]
          coordinates {
            (30,1.45)
            (30,1.40)
            (30,1.40)
            (30,1.40)
            (30,1.41)
            (30,1.41)
            (30,1.41)
            (30,1.42)
            (30,1.44)
            (30,1.44)
            (30,1.41)
            (30,1.43)
          };
        \end{axis}
      \end{tikzpicture}
      \caption{Sequential}
    \end{minipage}
\end{figure}


\end{document}